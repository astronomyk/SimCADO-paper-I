\section{Introduction}
\label{sec:Introduction}

Over the next decade the era of the extremely large telescopes will be begin. The European Extremely Large Telescope \citep[ELT, ][]{eelt} will provide astronomers with the increase in resolution and sensitivity needed to solve many of the outstanding questions of modern day astronomy. 

With a 39\,m primary mirror consisting of 798 individually steerable 1.45\,m hexagonal mirror segments and a fully deformable quaternary mirror, the ELT will be capable of providing diffraction-limited imaging with the help of the adaptive optics (AO) modules. This corresponds to a core full-width half maximum (FWHM) for the point spread function (PSF) in the J-band ($1.2\,\micron$) of $8\,\mathrm{mas}$ and $14\,\mathrm{mas}$ in the Ks-band ($2.16\,\micron$). Both single- and multi-conjugate modes for the adaptive optics will be possible. Laser guide stars (LGS) will be available to ensure that the ELT will always be able to provide AO-assisted observations. 

As the first-light wide-field imaging camera for the ELT, MICADO -- the \textbf{M}ulti-AO \textbf{I}maging \textbf{CA}mera for \textbf{D}eep \textbf{O}bservations \citep{micado} -- will take advantage of the ELT's near-infrared optimised design to provide images at the diffraction limit. MICADO will provide a wide-field imaging mode and a zoom (narrow-field) mode with 4\,mas/pixel and 1.5\,mas/pixel plate-scales respectively. The detector plane will consist of nine $4096\times 4096$ detector chips, allowing MICADO to cover a Field of View (FOV) of $\sim55\arcsec$ by $50\arcsec$ in the wide-field mode and $21\arcsec$ by $19\arcsec$ in the high resolution mode.

MICADO will also contain a series of additional modes, including: a long-slit spectrographic mode with a spectral resolution of up to $R\sim20\,000$ for point sources and $R\sim8000$ for extended sources, and a high-contrast imaging mode.

As the scale and complexity of telescopes and instruments increases, so too does the importance of accurately being able to predict the performance of these systems. More and more emphasis is being placed on developing simulation software to model new instruments before they enter the construction phase. As part of the development of the MICADO instrument, the MICADO Data Flow System (DFS) team has been tasked with creating a tool to simulate raw detector read-out images based on the current designs of the ELT and MICADO. Here we present SimCADO, the instrument data simulator for MICADO. 

SimCADO combines the most recent data from the other work packages in the consortium to allow the user to simulate the above mentioned raw data frames as will be produced by the ELT/MICADO optical system. Originally conceived as a tool to aid the development of the data reduction pipeline, SimCADO has also found use among the science team as a way of conducting feasibility studies for various future observations.

% In this paper we briefly introduce the SimCADO package and show that it is a useful tool for producing not only accurate simulated images for the MICADO/ELT optical system, but also for any other optical train, such as for HAWK-I on UT4 at the VLT. It should be noted that the design of SimCADO is described in detailed in \citet{leschinski2016}. Aspects which have been updated in the mean time are detailed in this paper. For all other design aspects the reader should still consult \citet{leschinski2016}. This paper is organised in the following way: Sect.~\ref{sec:scope} delves briefly into the motivation behind creating SimCADO as well as giving an overview of the scope of the project. The physical effects that SimCADO models are described in Sect.~\ref{sec:simcado}. A description of how we validated SimCADO by comparing simulated VLT/HAWK-I read-out frames to real images from the ESO archive can be found in Sect.~\ref{sec:hawki} and predictions for the sensitivity of the ELT/MICADO system are presented in Sect.~\ref{sec:predictions}. A discussion of the results, assumptions and issues with the simulated images is presented in Sect.~\ref{sec:discussion}.

% ..todo:
!!! Add a description of the paper here !!!