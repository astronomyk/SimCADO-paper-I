\section{SimCADO -- the Python package for simulating MICADO observations}
\label{sec:simcado}


\subsection{Code availability}
\label{subsec:code_availablity}

The astronomy community appears to be very much in line with the general trend in the scientific community towards using the Python language for scientific computing purposes.
As such the SimCADO package has been made for the Python ecosystem. Documentation for using SimCADO can be found on Read the Docs (\url{https://simcado.readthedocs.io}).
The package is available via the python package index (PyPI) and can be installed using the standard command:\\

\verb+pip install simcado+\\

The code is open source and publicly available for cloning at the author's GitHub repository: \url{https://github.com/astronomyk/SimCADO}. 

SimCADO is capable of simulating the readout images from the 9 HAWAII-4RG detectors in the MICADO focal plane and simulates the effect of various optical and mechanical elements, e.g: the field varying nature of the point spread function (PSF), atmospheric dispersion and spectral characteristics, residual field rotation from the derotator, transmission losses and thermal emission from all optical surfaces, the effects of non-common path aberrations on the system transmission, correlated and uncorrelated detector noise, detector linearity and saturation effects, photon shot noise, etc.
For a full list of effects in the package, the reader is directed to see the online documentation, or to refer to \citet{leschinski2016} for more details.
The online documentation also provides several tutorials and worked examples to help the user get started with the package.

While originally conceived as a way to mimic the telescope data environment for the purposes of developing the data reduction pipeline, SimCADO has also been adopted by the MICADO science team as a tool for conducting scientific and observational feasibility studies.
The results of some of these feasibility studies will be presented in future papers.
The remainder of this paper is dedicated to showing how we validated the accuracy of SimCADO simulations against raw HAWK-I/VLT images, and what that means for the predicted sensitivity of MICADO/ELT.
