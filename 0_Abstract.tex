\abstract{
MICADO will be the first-light near-infrared imaging camera at ESO's extremely large telescope. 
As part of the design activities for MICADO we have developed SimCADO: an instrument data simulator in the form of a Python package (\url{https://simcado.readthedocs.io/}).
To verify that the images produced by SimCADO are realistic in terms of both flux levels and spatially dependent optical artifacts we configured the software to mimic the HAWK-I/VLT-UT4 optical system. 
The characteristics of simulated HAWK-I images of two globular clusters compared very favourably with actual observations taken from the ESO archive. 
This confirmation of the software's accuracy allowed us to confidently make predictions about the point source sensitivity of MICADO at the ELT.
We find that the $5\sigma$ limits for a 5 hour SCAO observation in the J, H, and Ks filters are around 29.6$^m$, 29.3$^m$ 29.1$^m$ in the AB magnitude system. 
We also present simple observation horizons for a series of stellar spectral types for a range of total exposure time and in the J and Ks filters.
}

% context heading (optional)
% {} leave it empty if necessary
% {When the Extremely Large Telescope comes online at the end of 2025, MICADO will be the near infrared imaging camera available at first light. As part of the design activities for MICADO we have developed SimCADO: an instrument data simulator in the form of a Python package. SimCADO produces simulated data frames for MICADO's nine detector chips by applying the effects of each element along the optical train to a three dimensional ($x$, $y$, $\lambda$) description of the flux arriving from an astronomical object. Simulated images can be written to disk as FITS files and can be analysed by the standard suite of astronomical software.}
% aims heading (mandatory)
% {The main aims of this paper are two fold: to introduce the SimCADO software, and to provide preliminary detection limits for MICADO based on SimCADO simulations. By presenting these limits we also aim to raise awareness in the community of the future capabilities of MICADO and the current capabilities of SimCADO.}
% methods heading (mandatory)
% {To verify the accuracy of SimCADO we compared the characteristics of simulated images of globular clusters with images of the same clusters from the ESO archive. We then used a model of the current optical train design for MICADO to determine the sensitivity limits for MICADO at the ELT.}
% results heading (mandatory)
% {SimCADO is able to accurately reproduce the image characteristics of raw archival HAWK-I data to within the limits of the model of the UT4/HAWK-I optical train. Using the current design of MICADO, SimCADO finds detection limits in the J, H and Ks filters to be 28.7\m, 27.9\m and 27.3\m (Vega) respectively for a $5\sigma$  detection in a 5 hour observation. This leads to the ability to observe for the first time with a ground based observatory individual A0\,V stars at a distance of 4\,Mpc (i.e. in \object{Centaurus A}), and M9\,V stars in the \object{Large Magellanic Cloud}.}
% conclusions heading (optional), leave it empty if necessary 
% {}
